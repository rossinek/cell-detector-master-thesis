% Opcje klasy 'iithesis' opisane są w komentarzach w pliku klasy. Za ich pomocą
% ustawia się przede wszystkim język i rodzaj (lic/inz/mgr) pracy, oraz czy na
% drugiej stronie pracy ma być składany wzór oświadczenia o autorskim wykonaniu.
\documentclass[declaration,shortabstract]{iithesis}

\usepackage[utf8]{inputenc}

%%%%% DANE DO STRONY TYTUŁOWEJ
% Niezależnie od języka pracy wybranego w opcjach klasy, tytuł i streszczenie
% pracy należy podać zarówno w języku polskim, jak i angielskim.
% Pamiętaj o mądrym (zgodnym z logicznym rozbiorem zdania oraz estetyka) ręcznym
% złamaniu wierszy w temacie pracy, zwłaszcza tego w języku pracy. Użyj do tego
% polecenia \fmlinebreak.
\polishtitle    {Wymagający złamania wierszy\fmlinebreak tytuł pracy w~języku polskim}
\englishtitle   {English title}
\polishabstract {Polskie streszczenie}
\englishabstract{English abstract}
% w pracach wielu autorów nazwiska można oddzielić poleceniem \and
\author         {Artur Rosa}
% w przypadku kilku promotorów, lub konieczności podania ich afiliacji, linie
% w poniższym poleceniu można złamać poleceniem \fmlinebreak
\advisor        {dr Andrzej Łukaszewski}
%\date          {}                     % Data złożenia pracy
% Dane do oświadczenia o autorskim wykonaniu
%\transcriptnum {}                     % Numer indeksu
%\advisorgen    {dr. Jana Kowalskiego} % Nazwisko promotora w dopełniaczu
%%%%%

%%%%% WLASNE DODATKOWE PAKIETY
%
%\usepackage{graphicx,listings,amsmath,amssymb,amsthm,amsfonts,tikz}
%
%%%%% WŁASNE DEFINICJE I POLECENIA
%
%\theoremstyle{definition} \newtheorem{definition}{Definition}[chapter]
%\theoremstyle{remark} \newtheorem{remark}[definition]{Observation}
%\theoremstyle{plain} \newtheorem{theorem}[definition]{Theorem}
%\theoremstyle{plain} \newtheorem{lemma}[definition]{Lemma}
%\renewcommand \qedsymbol {\ensuremath{\square}}
% ...
%%%%%

\begin{document}

%%%%% POCZĄTEK ZASADNICZEGO TEKSTU PRACY

\chapter{Wprowadzenie}
\section{Wstęp}



\section{Background}


\subsection{Jakich danych i do czego potrzebują biolodzy}

\subsection{Pozyskiwanie danych}





\section{Wykrywanie komórek}


\subsection{Obraz wejściowy}
\subsection{Metody wykrywania kształtów glistopodobnych}
\subsection{Użyta metoda}

Zalety i wady różnych kanałów obrazów wejściowych

Preprocessing (shape index)

Szkieletonizacja

Wydzielanie kręgosłupów jako komórek

Rozwiązywanie konfliktów (podział komórek)

Poprawianie końcówek (dociąganie do krawędzi)

\section{Śledzenie komórek w czasie}





\section{Interfejs użytkownika}






\section{Opis implementacji}

\subsection{Kod źródłowy}
\subsection{Kompilacja i uruchomienie}

\subsection{Struktury danych}
\subsection{Architektura}
\subsection{Opis wykorzystanego API ImageJ}
\subsection{Błędy w implementacji ImageJ i sposoby na ich obejście}
\subsection{Opis sposobu dalszego rozwoju, interfejsy}







\section{Zakończenie}

\subsection{Podsumowanie}
\subsection{Ograniczenia wynikające z zastosowanych metod}
\subsection{Dalszy rozwój}



%%%%% BIBLIOGRAFIA


%\bibitem{example} \ldots

\begin{thebibliography}{1}
\bibitem{example}
  Example Women,
  \emph{A Document Preparation System}.
  Addison Wesley, Massachusetts,
  2nd Edition,
  1994.
\end{thebibliography}

\end{document}
