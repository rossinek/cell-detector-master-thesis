% Opcje klasy 'iithesis' opisane są w komentarzach w pliku klasy. Za ich pomocą
% ustawia się przede wszystkim język i rodzaj (lic/inz/mgr) pracy, oraz czy na
% drugiej stronie pracy ma być składany wzór oświadczenia o autorskim wykonaniu.
\documentclass[declaration,shortabstract]{iithesis}

\usepackage[utf8]{inputenc}

%%%%% DANE DO STRONY TYTUŁOWEJ
% Niezależnie od języka pracy wybranego w opcjach klasy, tytuł i streszczenie
% pracy należy podać zarówno w języku polskim, jak i angielskim.
% Pamiętaj o mądrym (zgodnym z logicznym rozbiorem zdania oraz estetyka) ręcznym
% złamaniu wierszy w temacie pracy, zwłaszcza tego w języku pracy. Użyj do tego
% polecenia \fmlinebreak.
\polishtitle    {Wymagający złamania wierszy\fmlinebreak tytuł pracy w~języku polskim}
\englishtitle   {English title}
\polishabstract {Polskie streszczenie}
\englishabstract{English abstract}
% w pracach wielu autorów nazwiska można oddzielić poleceniem \and
\author         {Artur Rosa}
% w przypadku kilku promotorów, lub konieczności podania ich afiliacji, linie
% w poniższym poleceniu można złamać poleceniem \fmlinebreak
\advisor        {dr Andrzej Łukaszewski}
%\date          {}                     % Data złożenia pracy
% Dane do oświadczenia o autorskim wykonaniu
%\transcriptnum {}                     % Numer indeksu
%\advisorgen    {dr. Jana Kowalskiego} % Nazwisko promotora w dopełniaczu
%%%%%

%%%%% WLASNE DODATKOWE PAKIETY
%
%\usepackage{graphicx,listings,amsmath,amssymb,amsthm,amsfonts,tikz}
%
%%%%% WŁASNE DEFINICJE I POLECENIA
%
%\theoremstyle{definition} \newtheorem{definition}{Definition}[chapter]
%\theoremstyle{remark} \newtheorem{remark}[definition]{Observation}
%\theoremstyle{plain} \newtheorem{theorem}[definition]{Theorem}
%\theoremstyle{plain} \newtheorem{lemma}[definition]{Lemma}
%\renewcommand \qedsymbol {\ensuremath{\square}}
% ...

\newcommand{\image}{\mathbf{I}}
%%%%%

\begin{document}

%%%%% POCZĄTEK ZASADNICZEGO TEKSTU PRACY

\chapter{Wprowadzenie}

%%%%%%%%%%%%%%%%%%%%%%%%%%%%%%%%%%%%%%%%%%%%%%%%%%%%%%%%%%%%%%%%%%%%%%%%%%%%%%%%%%%%%%%%%%%

\section{Wstęp}

% TODO Wstęp


%%%%%%%%%%%%%%%%%%%%%%%%%%%%%%%%%%%%%%%%%%%%%%%%%%%%%%%%%%%%%%%%%%%%%%%%%%%%%%%%%%%%%%%%%%%

\section{Background}

% TODO Background

\subsection{Jakich danych i do czego potrzebują biolodzy}

\subsection{Pozyskiwanie danych}


%%%%%%%%%%%%%%%%%%%%%%%%%%%%%%%%%%%%%%%%%%%%%%%%%%%%%%%%%%%%%%%%%%%%%%%%%%%%%%%%%%%%%%%%%%%

\chapter{Wykrywanie i śledzenie komórek}

\section{Opis problemu}
\label{sec:problem}

\subsection{Obrazy wejściowe}
\label{sec:input-images}

% TODO Obrazy wejściowe

\subsection{Pożądany efekt}
\label{sec:output}
% TODO Pożądany efekt

\section{Powiązane prace}

% TODO Powiązane prace

\section{Wykrywanie komórek}

\subsection{Wstęp}

Problem opisany w sekcji \ref{sec:problem} zdefiniowany jest dla nagrań spod mikroskopu. Postanowiłem jednak najpierw rozwiązać podobny problem, ale zdefiniowany dla pojedynczego obrazu. Rozwiązanie tego problemu mogłoby z łatwością zostać uogólnione na stos obrazów (nagranie). W tym rozdziale opiszę rozwiązanie uproszczonego problemu: oznaczanie komórek widocznych na pojedynczym obrazie.


\subsection{Dane wejściowe}

% TODO Dane wejściowe - user input itp.
Oznaczenie wszystkich komórek widocznych na obrazie można rozłożyć na dwa osobne problemy:
\begin{enumerate}
  \item Określenie liczby oraz lokalizacji poszczególnych komórek
  \item Oznaczenie kręgosłupów komórek.
\end{enumerate}

W niniejszej pracy zdecydowałem się nie rozwiązywać automatycznie pierwszego problemu. Zamiast tego użytkownik zobowiązany jest ręcznie zaznaczyć dokładnie jeden punkt wewnątrz każdej komórki widocznej na obrazie. Wymóg ten dotyczy tylko pierwszej klatki nagrania, co opiszę dokładniej w dalszej części pracy (\ref{sec:cell-tracking}).

% TODO z którego kanału?
Danymi wejściowymi są zatem obraz $\image$ oraz zbiór punktów $\mathbf{P}$ lokalizujących komórki.



\subsection{Wstępne przetwarzanie obrazu}

% TODO Wstępne przetwarzanie obrazu

Preprocessing (shape index), ale też zalety i wady różnych kanałów obrazów wejściowych

\subsection{Szkieletyzacja i wstępna detekcja komórek}

% TODO Szkieletyzacja i wstępna detekcja komórek

Wydzielanie kręgosłupów jako komórek

\subsection{Rozwiązywanie konfliktów}

% TODO Rozwiązywanie konfliktów

\subsection{Korekta końcówek}

% TODO Korekta końcówek

Poprawianie końcówek (dociąganie do krawędzi)

\section{Śledzenie komórek w czasie}
\label{sec:cell-tracking}

% TODO Śledzenie komórek w czasie
Opis jednej iteracji + sprawdzenia czy komórka nie powinna zostać podzielona

\section{Interakcja ze strony użytkownika}

% TODO Interakcja ze strony użytkownika


%%%%%%%%%%%%%%%%%%%%%%%%%%%%%%%%%%%%%%%%%%%%%%%%%%%%%%%%%%%%%%%%%%%%%%%%%%%%%%%%%%%%%%%%%%%

\chapter{Opis implementacji}

% TODO Opis implementacji


\section{Kod źródłowy}
\section{Kompilacja i uruchomienie}

\section{Struktury danych}
\section{Architektura}
\section{Opis wykorzystanego API ImageJ}
\section{Błędy w implementacji ImageJ i sposoby na ich obejście}
\section{Opis sposobu dalszego rozwoju, interfejsy}





%%%%%%%%%%%%%%%%%%%%%%%%%%%%%%%%%%%%%%%%%%%%%%%%%%%%%%%%%%%%%%%%%%%%%%%%%%%%%%%%%%%%%%%%%%%

\chapter{Zakończenie}

% TODO Zakończenie


\section{Podsumowanie}
\section{Ograniczenia wynikające z zastosowanych metod}
\section{Dalszy rozwój}



%%%%% BIBLIOGRAFIA


%\bibitem{example} \ldots

\begin{thebibliography}{1}
\bibitem{example}
  Example Women,
  \emph{A Document Preparation System}.
  Addison Wesley, Massachusetts,
  2nd Edition,
  1994.
\end{thebibliography}

\end{document}
